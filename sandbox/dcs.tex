% dcs.tex

\section*{Document Control System}

Here we model a simple document control system in Z.
People who work together need to share their work, but there are many
opportunities for misunderstandings and confusion.  Errors can be
introduced when two people working on the same thing --- a file of
program code, for example --- make changes that conflict with each
other.  We can enlist the computer to help prevent such errors: this
is the purpose of a document control system.  Real examples include
SCCS (source code control system) and RCS (revision control
system).

Here is an excerpt from the informal description:

\begin{quote}
    If a user wants to check out a document in order to change
    the document and the user  has the permission to change it, 
    and nobody else is changing it at the moment, then that user
    may check the document out.

    As soon as a user has checked out a document for editing
    everyone else is disallowed from checking it out (of course people with
    read permission can read it).

    When the user is done editing the document, it should be
    checked in, allowing another user to check it out.
\end{quote}

Here is the Z model.  We begin by introducing two basic sets that
hold everything of interest in this universe, namely people and
documents:

\begin{zed} [PERSON, DOCUMENT] \end{zed}
Some people have permission to change particular documents.  We can model
that as a relation on documents and people:

\begin{axdef}
	permission: DOCUMENT \rel PERSON
\end{axdef}
This relation is just a set of pairs of the form $(document, person)$.
For example, Doug can change the specification, Aki and Doug can
change the design, and Aki and Phil can change the code.

\begin{axdef}
	doug, aki, phil: PERSON \\
	spec, design, code: DOCUMENT
\end{axdef}
\begin{zed}
	permission = \{ (spec, doug), (design, doug), (design, aki), (code, aki), (code, phil) \}
\end{zed}
The state of the system is merely another relation of the same type,
this one saying which documents are actually checked out to whom.   The
central requirement is that a document can only be checked out to
one person at a time, so in this case the relation is a function:
it associates each object in the domain with a single
object in the range.  

\begin{schema}{Documents}
	checked\_out: DOCUMENT \pfun PERSON
\where
	checked\_out \subseteq permission
\end{schema}
Note that $checked\_out$ is a partial function, indicated by the stroke
through the arrow $\pfun$.  This means that some documents might not be
checked out to anybody.  The predicate says that documents can only be
checked out to people who have permission to change them.

A possible state of our system occurs when Doug has checked out the
specification and design, and Phil has checked out the code.

\begin{zed} 
	checked\_out = \{ (design, doug), (spec, doug), (code, phil) \}
\end{zed}

We need two operations that change the state, $CheckOut$ and $CheckIn$.
Here is $CheckOut$.  It has two input parameters, the person $p?$ and the document $d?$.

\begin{schema}{CheckOut}
	\Delta Documents \\
	p?: PERSON \\
	d?: DOCUMENT
\where
	d? \notin \dom checked\_out \\
	(d?, p?) \in permission \\
	checked\_out' = checked\_out \cup \{(d?, p?)\}
\end{schema}
$CheckOut$ has two preconditions.  They are the predicates that contain
no primed ``after'' variables.  First, document $d?$ can't already be checked
out: it can't be in the domain of $checked\_out$.  Moreover, the person
doing the checking out needs permission:  $(d?, p?)$ must belong to
$permission$.  If the preconditions are satisfied we add the pair $(d?,p?)$
to $checked\_out$; this prevents anyone else from checking out $d?$.

We must account for cases where the preconditions are not satisfied.
There are two preconditions, so there must be two such cases.
$CheckedOut$ says that the document is already checked out: $d? \in
\dom checked\_out$.  $Unauthorized$ expresses that the person does not 
have permission: $(d?,p?) \notin permission$.  In both cases nothing
gets checked out and the state of the system does not change:
$\Xi~Documents$.

\begin{zed}
CheckedOut \defs [~ \Xi Documents; d?: DOCUMENT | d? \in \dom checked\_out ~]
\end{zed}

\begin{schema}{Unauthorized}
	\Xi Documents \\
	p?: PERSON \\
 	d?: DOCUMENT
\where
	(d?,p?) \notin permission
\end{schema}
The total operation $T\_CheckOut$ covers all three possibilities.

\begin{zed}
	T\_CheckOut \defs CheckOut \lor CheckedOut \lor Unauthorized
\end{zed}
This concludes our presentation.  This little study illustrates some 
typical features of Z models:

You can ignore details in order to focus on the aspects of the problem
you are interested in.  Here we concentrated on permissions and
keeping track of who has checked out what.  We did not model actually
copying documents back and forth between the central repository and
users' local directories.  We modelled the collections of documents
and users as fixed sets, and the permissions as a constant.  A real
document control system would have to provide some way to enter new
documents into the system and delete old ones, and there would also
have to be a way to assign and change permissions.  If we wished to 
model all that in Z we could represent people, documents, and
permissions as variables in the state schema instead of basic types
and global constants.

Models should be simple.  If you find yourself writing complicated functions
and quantified predicates, it is usually a sign that you are on the wrong
track.  Let the basic properties of sets, relations and functions do the
work for you.  The requirement that only one person at a time can check out
a document can be neatly expressed by a function.

The requirements about permissions can be represented by letting the
$checked\_out$ function be a subset of the $permission$ relation.  In
Z, functions are relations and relations are sets, so operators
defined for sets also apply to relations and functions, and you can
put them all together in the same expressions.  This is an advantage
of Z that you don't find in every formal notation.
